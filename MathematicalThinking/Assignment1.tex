\documentclass[12pt, letterpaper]{article}
\usepackage{graphicx}
\title{Introduction to Mathematical Thinking ASSIGNMENT 1}
\author{Jan Boije}
\date{January 2025}
\begin{document}
\maketitle

\begin{enumerate}
 \item Find two unambiguous (but natural sounding) sentences equivalent to the sentence The man saw
the woman with a telescope, the first where the man has the telescope, the second where the woman
has the telescope.

The man, looking through a telescope, saw the woman.

The man saw the woman, who had a telescope. 
 \item For each of the three ambiguous newspaper headlines I stated in the lecture, rewrite it in a way
that avoids the amusing second meaning, while retaining the brevity of a typical headline:
\begin{enumerate}
\item[a] Sisters reunited after ten years in checkout line at Safeway.

After ten years apart, sisters were reunited in the checkout line at Safeway.
\item[b] Large hole appears in High Street. City authorities are looking into it.

City authorities are looking into the fact that a large hole has appeared in High Street.

\item[c] Mayor says bus passengers should be belted.

Mayor says bus passengers should wear seat belts.
 \item[d] The following notice was posted on the wall of a hospital emergency room:
No head injury is too trivial to ignore.
Reformulate to avoid the unintended second reading. (The context for this sentence is so strong
that many people have difficulty seeing there is an alternative meaning.)

Head injury is never too trivial to ignore.
\end{enumerate}
 \item \textbf{You often see the following notice posted in elevators:
In case of fire, do not use elevator.
This one always amuses me. Comment on the two meanings and reformulate to avoid the unintended
second reading. (Again, given the context for this notice, the ambiguity is not problematic.)}
 \item Official documents often contain one or more pages that are empty apart from one sentence at the
bottom:
This page intentionally left blank.
Does the sentence make a true statement? What is the purpose of making such a statement?

What reformulation of the sentence would avoid any logical problems about truth? (Once again,
the context means that in practice everyone understands the intended meaning and there is no
problem. But the formulation of a similar sentence in mathematics at the start of the twentieth
century destroyed one prominent mathematician’s seminal work and led to a major revolution in
an entire branch of mathematics.)

this page intentionally left almost blank :)


 \item \textbf{Find (and provide citations for) three examples of published sentences whose literal meaning is
(clearly) not what the writer intended. [This is much easier than you might think. Ambiguity is
very common.]}
 \item Comment on the sentence “The temperature is hot today.” You hear people say things like this
all the time, and everyone understands what is meant. But using language in this sloppy way in
mathematics would be disastrous.
the temperature might be high (it is measured in degrees) but not hot.
 \item How would you show that not every number of the form N = (p1 · p2 · p3 · . . . · pn) + 1 is prime,
where p1, p2, p3, . . . , pn, . . . is the list of all prime numbers?
i could give an example showing when it is not true.
2*3+1 = 7 (prime)
2*3*5+1 = 31 (prime)
2*3*5*7+1 = 211 (prime)
2*3*5*7*11+1 = 2322 = 43*54
\end{enumerate}
JUST FOR FUN
\begin{enumerate}
\item Provide a context and a sentence within that context, where the word and occurs five times in
succession, with no other word between those five occurrences. (You are allowed to use punctuation.)

Regarding a sign above the pub 'The Rose and the crown'
You need to adjust the spacing: put a larger gap between 'Rose' and 'and', and 'and' and 'Crown'.

\item \textbf{Provide a context and a sentence within that context, where the words and, or, and, or, and occur
in that order, with no other word between them. (Again, you can use punctuation.)}
\end{enumerate}
\end{document}
