\documentclass[12pt, letterpaper]{article}
\usepackage{amsmath} % For mathematical symbols and environments
\usepackage{amssymb} % For additional symbols like \land
\title{Introduction to Mathematical Thinking ASSIGNMENT 2}
\author{Jan Boije}
\date{January 2025}
\begin{document}
\maketitle

\begin{enumerate}
 
\item \textbf{Simplify the following symbolic statements as much as you can, leaving your answer in the standard
symbolic form. (In case you are not familiar with the notation, I’ll answer the first one for you.)}
\begin{enumerate}
 
\item[(a)] $(\pi > 0) \land (\pi < 10) \xrightarrow{}  Answer: 0 < \pi < 10.$
\item[(b)] $(p \ge  7) \land (p < 12) \xrightarrow{}  Answer:$
\item[(c)] $(x > 5) \land (x < 7) \xrightarrow{}  Answer:$
\item[(d)] $(x < 4) \land (x < 6) \xrightarrow{}  Answer:$
\item[(e)] $(y < 4) \land (y^2 < 9) \xrightarrow{}  Answer:$
\item[(f)] $(x \ge 0) \land (x \le 0) \xrightarrow{}  Answer:$
\end{enumerate}
\item \textbf{Express each of your simplified statements from question 1 in natural English.}
\item \textbf{What strategy would you adopt to show that the conjunction $\phi_1 \land \phi_2 \land . . . \land \phi_n $ is true?}
\item \textbf{What strategy would you adopt to show that the conjunction $\phi_1 \land $\phi_2 \land . . . \land $\phi_n $ is false?}
\item \textbf{Simplify the following symbolic statements as much as you can, leaving your answer in a standard
symbolic form (assuming you are familiar with the notation):}
\begin{enumerate}
 
\item[(a)] (π > 3) ∨ (π > 10)
\item[(b)] (x < 0) ∨ (x > 0)
\item[(c)] (x = 0) ∨ (x > 0)
\item[(d)] (x > 0) ∨ (x ≥ 0)
\item[(e)] (x > 3) ∨ (x2 > 9)
\end{enumerate}
\item \textbf{Express each of your simplified statements from question 5 in natural English.}
\item \textbf{What strategy would you adopt to show that the disjunction φ1 ∨ φ2 ∨ . . . ∨ φn is true?}
\item \textbf{What strategy would you adopt to show that the disjunction φ1 ∨ φ2 ∨ . . . ∨ φn is false?}
\item \textbf{Simplify the following symbolic statements as much as you can, leaving your answer in a standard
symbolic form (assuming you are familiar with the notation):}
(a) ¬(π > 3.2)
(b) ¬(x < 0)
(c) ¬(x
2 > 0)
(d) ¬(x = 1)
(e) ¬¬ψ
\item \textbf{Express each of your simplified statements from question 9 in natural English.}
\item \textbf{Let D be the statement “The dollar is strong”, Y the statement “The Yuan is strong” and T
the statement “New US–China trade agreement signed”. Express the main content of each of the
following (fictitious) newspaper headlines in logical notation. (Note that logical notation captures
truth, but not the many nuances and inferences of natural language.) How would you justify and
defend your answers?}
(a) Dollar and Yuan both strong
(b) Yuan weak despite new trade agreement, but Dollar remains strong
(c) Dollar and Yuan can’t both be strong at same time.
1
(d) New trade agreement does not prevent fall in Dollar and Yuan
(e) US–China trade agreement fails but both currencies remain strong
\end{enumerate}
TWO TO THINK ABOUT AND DISCUSS WITH OTHER STUDENTS
\begin{enumerate}
\item \textbf{In US law, a trial verdict of “Not guilty” is given when the prosecution fails to prove guilt. This, of
course, does not mean the defendant is, as a matter of actual fact, innocent. Is this state of affairs
captured accurately when we use “not” in the mathematical sense? (i.e., Do “Not guilty” and “¬
guilty” mean the same?) What if we change the question to ask if “Not proven” and “¬ proven”
mean the same?}
\item \textbf{The truth table for ¬¬φ is clearly the same as that for φ itself, so the two expressions make identical
truth assertions. This is not necessarily true for negation in everyday life. For example, you might
find yourself saying “I was not displeased with the movie.” In terms of formal negation, this has
the form ¬(¬ pleased), but your statement clearly does not mean that you were pleased with the
movie. Indeed, it means something considerably less positive. How would you capture this kind of
use of language in the formal framework we have been looking at?}



\end{enumerate}
\end{document}
